\documentclass[12pt,a4paper,titlepage]{article}
\linespread{2.0}

\usepackage[T1]{fontenc}
\usepackage[utf8]{inputenc}
\usepackage{hyperref}
\usepackage{amsmath}

\begin{document}

\title{Technical Report on Convolutional Neural Networks}
\author{Sebastien Champoux
\\ 40133449
\\ Concordia University
\\ ENGR 411 Technical Report
}
\maketitle

\begin{abstract}
I'm baby semiotics salvia literally +1 af coloring book woke banh mi gentrify organic wolf 8-bit tote bag chartreuse chicharrones. Helvetica cardigan next level art party kickstarter, vexillologist banh mi master cleanse actually street art tumeric poke whatever.
\end{abstract}
\clearpage

\tableofcontents
\clearpage

\listoffigures
\clearpage

\section{Introduction}
[Talk about the fact that we are talking about image classification specifically.]

\section{Convolutional neural networks}
Convolutional neural networks are a variation of multilayer perceptrons, a common class of neural networks. Therefore, it makes sense to start there.

\subsection{Multilayer perceptron}
\subsubsection{Structure}
A classical concrete application of a neural network is the identification of handwritten digits. This application is commonly used as an example thanks to the existence of the MNIST Database (\textit{Modified National Institute of Standards and Technology database}), a database of 70,000 labelled pictures of handwritten digits freely available for machine learning \cite{lecun_mnist_1998}. To simplify the presentation of multilayer perceptrons below, this application will be used to illustrate how the different components of a neural network can work together to achieve this application.

A multilayer perceptron is a class of neural network. As the name implies, it is made up of "neurons" that are connected together. A neuron is simply a real number in the range [0,1] that represents an activation; 0.0 meaning the neuron is not activated and 1.0 meaning the neuron is fully activated. The neurons are arranged in multiple layers, at least three: an input layer, an output layer, and at least one "hidden" layer, each one serving a different role.

Each neuron in the input layer represents a part of the total input provided to the perceptron. For the digits example, the input is a black and white image of a handwritten digit, of format 28*28 pixels. Therefore, the input layer would constitute of 784 neurons, each representing a pixel of the image, and their respective activation being the brightness value of the pixel (0.0 being black, 1.0 being white, and grey shades in-between) \cite{sanderson_but_2017}.

The output layer represents the output of the neural network. As the perceptron performs recognition and classification, each neuron in the output layer is one of the possible "choices" of what could be represented in the image. For the digits example, the output layer would contain ten neurons, one for each digit. The intended output of the neural network is for all but one of the output neurons to have a low activation, and one neuron having a very high activation, the latter being the network's best guess of what digit is represented in the image.

The hidden layers, in-between the input and output layers, serve to improve the performance of the neural network. These layers enable the network to recognize patterns within the image, therefore facilitating the recognition of complex forms. For instance, the digit "8" is made up of two small loops one over the other. In theory, the middle layers could attempt to recognize these shapes in the image to make a decision whether the digit represented is an eight. In practice, neural networks create their own patterns during the training process, which are often very abstract and a far cry of what a human would consider logical \cite{sanderson_gradient_2017}.

In a standard multilayer perceptron, with the exception of neurons in the input layer, every neuron from a layer is connected to every neuron from the previous layer. The activation of a neuron affects the activation of every neuron connected to it ; in other words, with the exception of the input layer, neurons are functions of the activation of the neurons that preceed them. The connection in-between two layers can be weighted, to increase or decrease how much the activation of a neuron impacts the connection of another neuron.

A bias can also be introduced into the equation. The addition of a bias can help improve the results by requiring a very high, or very low, activation in parts of the input layer before activating some specific neurons in the following layers \cite{sanderson_but_2017}.

All of these elements can be summarized into a vector-matrix multiplication. Below is the equation to compute the activation of the neurons in hidden layer 1, based on the activation of neurons from the input layer (layer 0). \(a_i^{(0)}\) represents the activation of neuron \(i\) from layer 0, \(w_{i,j}\) the weight of the connection between neuron \(i\) from the input layer and neuron \(j\) in layer 1, and \(b_i\) the bias applied to the weighted sum of the incoming connections from all of layer 0 to neuron \(i\) in layer 1. This output vector is processed through a function, usually the sigmoid function, to squish the results into the range \([0,1]\). This output neuron represents the activation of every neuron in layer 1. This process can then be repeated for the subsequent layers, using the activation vector of layer 1 as input \cite{sanderson_but_2017}.

\begin{displaymath}
	\sigma
	\left(
	\begin{bmatrix}
		w_{0,0} & w_{0,1} & \cdots & w_{0,n}\\
		w_{1,0} & w_{1,1} & \cdots & w_{1,n}\\
		\vdots & \vdots & \ddots & \vdots\\
		w_{m,0} & w_{m,1} & \cdots & w_{m,n}
	\end{bmatrix}
	\begin{bmatrix}
		a_{0}^{(0)}\\
		a_{1}^{(0)}\\
		\vdots\\
		a_{n}^{(0)}
	\end{bmatrix}
	+
	\begin{bmatrix}
		b_{0}\\
		b_{1}\\
		\vdots\\
		b_{n}
	\end{bmatrix}
	\right)
	=
	\begin{bmatrix}
		a_{0}^{(1)}\\
		a_{1}^{(1)}\\
		\vdots\\
		a_{n}^{(1)}
	\end{bmatrix}
\end{displaymath}

As demonstrated above, a neural network boils down to fairly simple, albeit large, linear algebraic functions. For this reason, neural networks are often computed by graphical processing units (GPU) instead of regular CPUs, as the formers are optimized for processing vector-matrices computations \cite{salter_cart_2022}.

\subsubsection{Deep learning, gradient descent and backpropagation}
Once the structure of the neural network has been laid down, it must be tuned so that it can recognize and classify accurately what is presented to it. This tuning process consists of adjusting the hundreds of weights and biases in-between neurons, in order for a given input to trigger the appropriate activations in the output layer with an acceptable rate of accuracy. It goes without saying that this is too tedious and intricate to be performed by humans. Therefore, the neural network "trains itself" with the help of two algorithms, gradient descent and backpropagation. \cite{ibm_cloud_education_what_2020}.

To train a neural network, it is necessary to have a training set on-hand. A training set is a set of contents (images, videos, audio clips or some other medium), each item of which is labelled with the expected classification. As discussed earlier, the MNIST database is a good example of a training set: it is a set of 70,000 images of hand-drawn digits, each image appropriately labelled with the represented digit \cite{lecun_mnist_1998}. Another example is the database created by Google's ReCAPTCHA service. This service helps to protect websites from bots by asking web users to identify objects within images. By doing so, human users of ReCAPTCHA are also, unknowingly, contributing to the elaboration of training sets for AI training \cite{maruzani_are_2021}.

To begin the training process, the different weights and biases of the neural network are initialised with random values. The training set is then processed by the neural network. Unsurprisingly, the results will be very poor at this stage. The network computes the accuracy (or lack thereof) of its results with a cost function. This function receives, as input, the different weights and biases in the neural network, and returns a single real number, the cost of the network's errors. This cost is computed by averaging the error cost of each element in the training set. The latter, the error cost for a single example, is the sum of the squared difference between the actual activation of each neuron in the output layer with the expected activation of this neuron.

Error cost of a single element in the training set:
\begin{displaymath}
	\sum_{i=0}^{n} (a_{i}^{(o)} - a_{i}^{(p)})^{2}
\end{displaymath}

This cost function expresses how accurate the results given by the neural network are. The higher the error cost, the greater the difference between the expected and computed results and thus, the less accurate the network is. The objective is to adjust the weights and biases in the neural network to decrease this error cost as close as possible to 0 over the training set.

To reduce this cost function, gradient descent will be used. Gradient descent is an optimization algorithm to find a local minimum of a function. It consists of computing the gradient of a function (noted \(\nabla C\) - a vector in the direction of the steepest increase), moving by a factor of the negative of this vector (\(-\nabla C\)), and repeating until a local minimum is reached \cite{sanderson_gradient_2017}.

The algorithm to efficiently compute the gradient of a neural network is backpropagation. This algorithm computes the adjustments that should be made to every weight and bias in the neural network in order to minimize the error cost. For each element in the training set, the algorithm starts from the output layer, and compares the activation of the output layer's neurons over a given example of the training set, to the expected result for this training set. Then, it computes by how much each weight and bias between the output layer and its previous layer should be adjusted in order to get closer to the expected result. Connections (weights) that improve the results of the network (decrease the error cost) should be strenghtened, and conversely, connections that increase this error cost should be weakened. These adjustments should be proportional to how much each connection impacts the error cost. This algorithm is repeated recursively for each layer in the neural network, then the adjustments for each example in the training set are averaged to get an adjustment vector for the entire training set \cite{sanderson_gradient_2017}.

[NOTATION in video 4]

At this point, it should be noted that computing the error cost and gradient descent of a neural network over a training set of tens of thousands of examples is computationaly expensive. For this reason, a more performant technique, stochastic gradient descent, is commonly used. The algorithm is the same, but instead of computing the entire training set, the training set is subdivided into multiple smaller random subsets of training examples. Then, the gradient descent vector is computed for every subset of the training set, and these vectors are applied iteratively to gradually improve the results of the network \cite{sanderson_gradient_2017}.

[FORMULA?]

\subsection{Convolutions}

\subsection{Using image convolutions in a neural network}

\section{Image analysis with convolutional neural networks}

\subsection{Concrete applications of CNNs}

\section{Favorable characteristics of CNNs for image recognition}

\section{Conclusion}

\clearpage
\begin{flushleft}
\bibliographystyle{plain}
\bibliography{references}
\end{flushleft}

\end{document}